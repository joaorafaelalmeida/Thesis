\chapter{Introdução}%
\label{chapter:introduction}


\begin{introduction}
\end{introduction}

In today’s rapidly evolving digital landscape, security has become a crucial aspect of software development, especially as cyberattacks grow in sophistication and frequency. To address these challenges, the DevSecOps approach has emerged, integrating security practices into the traditional DevOps pipeline. While DevOps focuses on automating and streamlining the development and operations processes, DevSecOps adds an essential layer of security throughout the Continuous Integration and Continuous Delivery (CI/CD) pipelines. This guarantees the identification and removal of vulnerabilities during the initiation of the development cycle. However, integrating automated security tools effectively into these pipelines remains a challenge for many organizations, often caused by the concern to disrupt the development workflows or generating excessive complexity.
This research aims to explore and develop a vulnerability management system that can be seamlessly integrated into CI/CD pipelines. By automating the use of Static Application Security Testing (SAST) and Dynamic Application Security Testing (DAST) tools, the proposed framework will develop a continuous monitoring, detection, and reporting of security vulnerabilities in Web Applications. The effectiveness of such work will be validated by using a real-world use case—the EHDEN portal, the one responsible for an easier access to health databases, where data protection and security are paramount. Through this work, the goal is to provide an efficient, scalable solution that balances security needs with the agility and speed required in modern software development environments.
Incorporating security into the software development lifecycle is no longer optional, it is a necessity. As Web Applications handle increasingly more sensitive data, such as healthcare information, a robust and automated security framework becomes essential. This research aims to bridge the gap between DevOps and security, ensuring that organizations can maintain high security standards without compromising the speed and efficiency of their development processes.




\section{Motivation}

In recent years, the implementation of DevOps practices has become essential to accelerate software development and IT operations, providing an improvement in efficiency and collaboration between teams. Moreover, the growing complexity of cyberattacks has highlighted the importance of integrating security from the earliest stages of the development cycle, culminating in the concept of DevSecOps. This concept advocates the inclusion of security practices throughout the CI/CD pipelines, aiming to identifying and mitigating vulnerabilities promptly. Nonetheless, many organizations still face challenges in effectively incorporating automated security tools without disrupting the development flow.

The efficiency of these vulnerability management systems became evidently more important as the development of web applications continues to grow. Security vulnerabilities may compromise the integrity of data and operations at risk as well as compromise user trust. SAST and DAST tools have the potential to automate the detection of security flaws. However, only one of these tools tends to be used, or, when combined, it is not effectively integrated into CI/CD pipelines. The fragmentation between these tools and the development flow can result in undetected security vulnerabilities, exposing applications to risks.

Additionally, the increasing volume of discovered vulnerabilities requires a centralized and automated reporting system. This process shall provide actionable feedback to developers without delaying the development process. The creation of clear dashboards and reports is crucial to ensure that vulnerabilities are identified and resolved quickly. Such need is particularly critical in sensitive sectors, such as healthcare, where data protection and regulatory compliance are essential. One example of such is the implementation of a vulnerability management system in the EHDEN portal, which facilitates access to health databases, a key element to ensuring patient privacy and data integrity.

Hence the investigation of an efficient way to integrate and automate vulnerability management within CI/CD pipelines is so significant, leading to the improvement of the security of Web Applications. The development of a framework that combines the best security practices with tools for detecting both static and dynamic vulnerabilities is not only an important step toward ensuring continuous security but also aligns development processes with emerging security standards in the DevSecOps world.

\section{Goal}
%Write something here preenting the researche question

How can a vulnerability management system be effectively integrated into CI/CD pipelines to detect, monitor, and report security vulnerabilities in Web Applications without disrupting development workflows?



%My proposals (but i do not like them, just to think a bit better):
%How can a vulnerability system detect and report security issues in CI/CD pipelines?
%How can CI/CD pipelines monitor and report web app vulnerabilities using automated tools?
%How can CI/CD benefit from automatised vulnerability detecting systems?
